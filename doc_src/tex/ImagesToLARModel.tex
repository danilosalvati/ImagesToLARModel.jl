\documentclass[11pt,oneside]{article}	%use"amsart"insteadof"article"forAMSLaTeXformat
\usepackage{geometry}		%Seegeometry.pdftolearnthelayoutoptions.Therearelots.
\geometry{letterpaper}		%...ora4paperora5paperor...
%\geometry{landscape}		%Activateforforrotatedpagegeometry
%\usepackage[parfill]{parskip}		%Activatetobeginparagraphswithanemptylineratherthananindent
\usepackage{graphicx}				%Usepdf,png,jpg,orepswithpdflatex;useepsinDVImode
								%TeXwillautomaticallyconverteps-->pdfinpdflatex		
\usepackage{amssymb}
\usepackage[colorlinks]{hyperref}

\usepackage{array}
\newcolumntype{P}[1]{>{\raggedright\arraybackslash}p{#1}}

%----macros begin---------------------------------------------------------------
\usepackage{color}
\usepackage{amsthm}

\def\conv{\mbox{\textrm{conv}\,}}
\def\aff{\mbox{\textrm{aff}\,}}
\def\E{\mathbb{E}}
\def\R{\mathbb{R}}
\def\Z{\mathbb{Z}}
\def\tex{\TeX}
\def\latex{\LaTeX}
\def\v#1{{\bf #1}}
\def\p#1{{\bf #1}}
\def\T#1{{\bf #1}}

\def\vet#1{{\left(\begin{array}{cccccccccccccccccccc}#1\end{array}\right)}}
\def\mat#1{{\left(\begin{array}{cccccccccccccccccccc}#1\end{array}\right)}}

\def\lin{\mbox{\rm lin}\,}
\def\aff{\mbox{\rm aff}\,}
\def\pos{\mbox{\rm pos}\,}
\def\cone{\mbox{\rm cone}\,}
\def\conv{\mbox{\rm conv}\,}
\newcommand{\homog}[0]{\mbox{\rm homog}\,}
\newcommand{\relint}[0]{\mbox{\rm relint}\,}

%----macros end-----------------------------------------------------------------

\title{ImagesToLARModel, a tool for creation of three-dimensional models from a stack of images}
\author{Danilo Salvati}
%\date{}							%Activatetodisplayagivendateornodate

\begin{document}
\maketitle
%\nonstopmode

\begin{abstract}
This is the abstract (we will use LAR~\cite{cclar-proj:2013:00})

\end{abstract}

\newpage
\tableofcontents
\newpage

%-------------------------------------------------------------------------------
%===============================================================================
\section{Introduction}\label{sec:intro}
%===============================================================================

%-------------------------------------------------------------------------------
%-------------------------------------------------------------------------------
\section{Exporting the library}
%-------------------------------------------------------------------------------

@O src/ImagesToLARModel.jl
@{module ImagesToLARModel
"""
Main module for the library. It starts conversion
taking configuration parameters
"""
require(string(Pkg.dir("ImagesToLARModel/src"), "/imagesConvertion.jl"))

import JSON
import ImagesConvertion

using Logging

export convertImagesToLARModel

function loadConfiguration(configurationFile)
  """
  load parameters from JSON file
  
  configurationFile: Path of the configuration file
  """

  configuration = JSON.parse(configurationFile)
  
  DEBUG_LEVELS = [DEBUG, INFO, WARNING, ERROR, CRITICAL]

  return configuration["inputDirectory"], configuration["outputDirectory"], configuration["bestImage"],
        configuration["nx"], configuration["ny"], configuration["nz"],
        DEBUG_LEVELS[configuration["DEBUG_LEVEL"]]

end

function convertImagesToLARModel(configurationFile)
  """
  Start convertion of a stack of images into a 3D model
  loading parameters from a JSON configuration file
  
  configurationFile: Path of the configuration file
  """
  inputDirectory, outputDirectory, bestImage, nx, ny, nz, DEBUG_LEVEL = loadConfiguration(open(configurationFile))
  convertImagesToLARModel(inputDirectory, outputDirectory, bestImage, nx, ny, nz, DEBUG_LEVEL)
end

function convertImagesToLARModel(inputDirectory, outputDirectory, bestImage,
                                 nx, ny, nz, DEBUG_LEVEL = INFO)
  """
  Start convertion of a stack of images into a 3D model
  
  inputDirectory: Directory containing the stack of images
  outputDirectory: Directory containing the output
  bestImage: Image chosen for centroids computation
  nx, ny, nz: Border dimensions (Possibly the biggest power of two of images dimensions)
  DEBUG_LEVEL: Debug level for Julia logger. It can be one of the following:
    - DEBUG
    - INFO
    - WARNING
    - ERROR
    - CRITICAL
  """
  # Create output directory
  try
    mkpath(outputDirectory)
  catch
  end

  Logging.configure(level=DEBUG_LEVEL)
  ImagesConvertion.images2LARModel(nx, ny, nz, bestImage, inputDirectory, outputDirectory)
end
end
@}

@O src/imagesConvertion.jl
@{module ImagesConvertion

require(string(Pkg.dir("ImagesToLARModel/src"), "/generateBorderMatrix.jl"))
require(string(Pkg.dir("ImagesToLARModel/src"), "/pngStack2Array3dJulia.jl"))
require(string(Pkg.dir("ImagesToLARModel/src"), "/lar2Julia.jl"))
require(string(Pkg.dir("ImagesToLARModel/src"), "/model2Obj.jl"))

import GenerateBorderMatrix
import PngStack2Array3dJulia
import Lar2Julia
import Model2Obj

import JSON

using PyCall
@@pyimport scipy.sparse as Pysparse

using Logging

export images2LARModel

"""
This is main module for converting a stack
of images into a 3d model
"""

function images2LARModel(nx, ny, nz, bestImage, inputDirectory, outputDirectory)
  """
  Convert a stack of images into a 3d model
  """

  info("Starting model creation")

  numberOfClusters = 2 # Number of clusters for
                       # images segmentation

  imageWidth, imageHeight = PngStack2Array3dJulia.getImageData(string(inputDirectory,bestImage))
  imageDepth = length(readdir(inputDirectory))

  # Computing border matrix
  info("Computing border matrix")
  try
    mkdir(string(outputDirectory, "BORDERS"))
  catch
  end
  borderFilename = GenerateBorderMatrix.getOriented3BorderPath(string(outputDirectory, "BORDERS"), nx, ny, nz)

  # Starting images convertion and border computation
  info("Starting images convertion")
  startImageConvertion(inputDirectory, bestImage, outputDirectory, borderFilename,
                       imageHeight, imageWidth, imageDepth,
                       nx, ny, nz,
                       numberOfClusters)

end


function startImageConvertion(sliceDirectory, bestImage, outputDirectory, borderFilename,
                              imageHeight, imageWidth, imageDepth,
                              imageDx, imageDy, imageDz,
                              numberOfClusters)
  """
  Support function for converting a stack of images into a model

  sliceDirectory: directory containing the image stack
  imageForCentroids: image chosen for centroid computation
  """

  info("Moving images into temp directory")
  try
    mkdir(string(outputDirectory, "TEMP"))
  catch
  end

  tempDirectory = string(outputDirectory,"TEMP/")

  newBestImage = PngStack2Array3dJulia.convertImages(sliceDirectory, tempDirectory, bestImage)

  # Create clusters for image segmentation
  info("Computing image centroids")
  debug("Best image = ", bestImage)
  centroidsCalc = PngStack2Array3dJulia.calculateClusterCentroids(tempDirectory, newBestImage, numberOfClusters)
  debug(string("centroids = ", centroidsCalc))

  try
    mkdir(string(outputDirectory, "BORDERS"))
  catch
  end
  debug(string("Opening border file: ", "border_", imageDx, "-", imageDy, "-", imageDz, ".json"))
  boundaryMat = getBorderMatrix(string(outputDirectory,"BORDERS/","border_", imageDx, "-",
                                       imageDy, "-", imageDz, ".json"))
  beginImageStack = 0
  endImage = beginImageStack

  info("Converting images into a 3d model")
  for zBlock in 0:(imageDepth / imageDz - 1)
    startImage = endImage
    endImage = startImage + imageDz
    info("StartImage = ", startImage)
    info("endImage = ", endImage)
    info(string("Start process convertion process ", zBlock))
    imageConvertionProcess(tempDirectory, outputDirectory,
                           beginImageStack, startImage, endImage,
                           imageDx, imageDy, imageDz,
                           imageHeight, imageWidth,
                           centroidsCalc, boundaryMat)
  end

  # TODO: add something for waiting all processes
  info("Merging obj models")
  Model2Obj.mergeObj(string(outputDirectory,"MODELS"))

end

function imageConvertionProcess(sliceDirectory, outputDirectory,
                                beginImageStack, startImage, endImage,
                                imageDx, imageDy, imageDz,
                                imageHeight, imageWidth,
                                centroids, boundaryMat)
  """
  Support function for converting a stack of image on a single
  independent process
  """

  info("Transforming png data into 3d array")
  theImage = PngStack2Array3dJulia.pngstack2array3d(sliceDirectory, startImage, endImage, centroids)

  centroidsSorted = sort(vec(reshape(centroids, 1, 2)))
  foreground = centroidsSorted[2]
  background = centroidsSorted[1]
  debug(string("background = ", background, " foreground = ", foreground))
  for xBlock in 0:(imageHeight / imageDx - 1)
    for yBlock in 0:(imageWidth / imageDy - 1)
      yStart = xBlock * imageDx
      xStart = yBlock * imageDy
      #xEnd = xStart + imageDx
      #yEnd = yStart + imageDy
      xEnd = xStart + imageDy
      yEnd = yStart + imageDx
      debug("***********")
      debug(string("xStart = ", xStart, " xEnd = ", xEnd))
      debug(string("yStart = ", yStart, " yEnd = ", yEnd))
      debug("theImage dimensions: ", size(theImage)[1], " ", size(theImage[1])[1], " ", size(theImage[1])[2])

      # Getting a slice of theImage array
      image = Array(Uint8, (convert(Int32, length(theImage)), convert(Int32, xEnd - xStart), convert(Int32, yEnd - yStart)))
      debug("image size: ", size(image))
      for z in 1:length(theImage)
        for x in 1 : (xEnd - xStart)
          for y in 1 : (yEnd - yStart)
            image[z, x, y] = theImage[z][x + xStart, y + yStart]
          end
        end
      end

      nx, ny, nz = size(image)
      chains3D = Array(Uint8, 0)
      zStart = startImage - beginImageStack
      for y in 0:(nx - 1)
        for x in 0:(ny - 1)
          for z in 0:(nz - 1)
            if(image[z + 1, x + 1, y + 1] == foreground)
              push!(chains3D, y + ny * (x + nx * z))
            end
          end
        end
      end

      if(length(chains3D) != 0)
        # Computing boundary chain
        debug("chains3d = ", chains3D)
        debug("Computing boundary chain")
        objectBoundaryChain = Lar2Julia.larBoundaryChain(boundaryMat, chains3D)
        debug("Converting models into obj")
        try
          mkdir(string(outputDirectory, "MODELS"))
        catch
        end
        # IMPORTANT: inverting xStart and yStart for obtaining correct rotation of the model
        outputFilename = string(outputDirectory, "MODELS/model-", xBlock, "-", yBlock, "_output_", startImage, "_", endImage)
        Model2Obj.writeToObj(imageDx, imageDy, imageDz, yStart, xStart, zStart, objectBoundaryChain, outputFilename)
      else
        debug("Model is empty")
      end
    end
  end
end

function getBorderMatrix(borderFilename)
  """
  TO REMOVE WHEN PORTING OF LARCC IN JULIA IS COMPLETED

  Get the border matrix from json file and convert it in
  CSC format
  """
  # Loading borderMatrix from json file
  borderData = JSON.parsefile(borderFilename)
  row = Array(Int64, length(borderData["ROW"]))
  col = Array(Int64, length(borderData["COL"]))
  data = Array(Int64, length(borderData["DATA"]))

  for i in 1: length(borderData["ROW"])
    row[i] = borderData["ROW"][i]
  end

  for i in 1: length(borderData["COL"])
    col[i] = borderData["COL"][i]
  end

  for i in 1: length(borderData["DATA"])
    data[i] = borderData["DATA"][i]
  end

  # Converting csr matrix to csc
  csrBorderMatrix = Pysparse.csr_matrix((data,col,row), shape=(borderData["ROWCOUNT"],borderData["COLCOUNT"]))
  denseMatrix = pycall(csrBorderMatrix["toarray"],PyAny)

  cscBoundaryMat = sparse(denseMatrix)

  return cscBoundaryMat

end

end
@}


@O src/generateBorderMatrix.jl
@{module GenerateBorderMatrix
"""
Module for generation of the boundary matrix
"""

type MatrixObject
  ROWCOUNT
  COLCOUNT
  ROW
  COL
  DATA
end


export computeOriented3Border, writeBorder, getOriented3BorderPath

require(string(Pkg.dir("ImagesToLARModel/src"), "/larUtils.jl"))

import LARUtils
using PyCall

import JSON

@@pyimport sys
unshift!(PyVector(pyimport("sys")["path"]), "") # Search for python modules in folder
# Search for python modules in package folder
unshift!(PyVector(pyimport("sys")["path"]), Pkg.dir("ImagesToLARModel/src"))
@@pyimport larcc # Importing larcc from local folder

# Compute the 3-border operator
function computeOriented3Border(nx, ny, nz)
  """
  Compute the 3-border matrix using a modified
  version of larcc
  """
  V, bases = LARUtils.getBases(nx, ny, nz)
  boundaryMat = larcc.signedCellularBoundary(V, bases)
  return boundaryMat

end

function writeBorder(boundaryMatrix, outputFile)
  """
  Write 3-border matrix on json file

  boundaryMatrix: matrix to write on file
  outputFile: path of the outputFile
  """

  rowcount = boundaryMatrix[:shape][1]
  colcount = boundaryMatrix[:shape][2]

  row = boundaryMatrix[:indptr]
  col = boundaryMatrix[:indices]
  data = boundaryMatrix[:data]

  # Writing informations on file
  outfile = open(outputFile, "w")

  matrixObj = MatrixObject(rowcount, colcount, row, col, data)
  JSON.print(outfile, matrixObj)
  close(outfile)

end

function getOriented3BorderPath(borderPath, nx, ny, nz)
  """
  Try reading 3-border matrix from file. If it fails matrix
  is computed and saved on disk in JSON format

  borderPath: path of border directory
  nx, ny, nz: image dimensions
  """

  filename = string(borderPath,"/border_", nx, "-", ny, "-", nz, ".json")
  if !isfile(filename)
    border = computeOriented3Border(nx, ny, nz)
    writeBorder(border, filename)
  end
  return filename

end
end
@}


@O src/lar2Julia.jl
@{module Lar2Julia
"""
larcc functions for Julia
"""
export larBoundaryChain, cscChainToCellList

import JSON

using Logging

function larBoundaryChain(cscBoundaryMat, brcCellList)
  """
  Compute boundary chains
  """

  # Computing boundary chains
  n = size(cscBoundaryMat)[1]
  m = size(cscBoundaryMat)[2]

  debug("Boundary matrix size: ", n, "\t", m)

  data = ones(Int64, length(brcCellList))

  i = Array(Int64, length(brcCellList))
  for k in 1:length(brcCellList)
    i[k] = brcCellList[k] + 1
  end

  j = ones(Int64, length(brcCellList))

  debug("cscChain rows length: ", length(i))
  debug("cscChain columns length: ", length(j))
  debug("cscChain data length: ", length(brcCellList))

  debug("rows ", i)
  debug("columns ", j)
  debug("data ", data)

  cscChain = sparse(i, j, data, m, 1)
  cscmat = cscBoundaryMat * cscChain
  out = cscBinFilter(cscmat)
  return out
end

function cscBinFilter(CSCm)
  k = 1
  data = nonzeros(CSCm)
  sgArray = copysign(1, data)

  while k <= nnz(CSCm)
    if data[k] % 2 == 1 || data[k] % 2 == -1
      data[k] = 1 * sgArray[k]
    else
      data[k] = 0
    end
    k += 1
  end

  return CSCm
end

function cscChainToCellList(CSCm)
  """
  Get a csc containing a chain and returns
  the cell list of the "+1" oriented faces
  """
  data = nonzeros(CSCm)
  # Now I need to remove zero element (problem with Julia nonzeros)
  nonzeroData = Array(Int64, 0)
  for n in data
    if n != 0
      push!(nonzeroData, n)
    end
  end

  cellList = Array(Int64,0)
  for (k, theRow) in enumerate(findn(CSCm)[1])
    if nonzeroData[k] == 1
      push!(cellList, theRow)
    end
  end
  return cellList
end
end
@}

@O src/larUtils.jl
@{module LARUtils
"""
Utility functions for extracting 3d models from images
"""
export ind, invertIndex, getBases

function ind(x, y, z, nx, ny)
    """
    Transform coordinates into linearized matrix indexes
    """
    return x + (nx+1) * (y + (ny+1) * (z))
  end


function invertIndex(nx,ny,nz)
  """
  Invert indexes
  """
  nx, ny, nz = nx + 1, ny + 1, nz + 1
  function invertIndex0(offset)
      a0, b0 = trunc(offset / nx), offset % nx
      a1, b1 = trunc(a0 / ny), a0 % ny
      a2, b2 = trunc(a1 / nz), a1 % nz
      return b0, b1, b2
  end
  return invertIndex0
end


function getBases(nx, ny, nz)
  """
  Compute all LAR relations
  """

  function the3Dcell(coords)
    x,y,z = coords
    return [ind(x,y,z,nx,ny),ind(x+1,y,z,nx,ny),ind(x,y+1,z,nx,ny),ind(x,y,z+1,nx,ny),ind(x+1,y+1,z,nx,ny),
            ind(x+1,y,z+1,nx,ny),ind(x,y+1,z+1,nx,ny),ind(x+1,y+1,z+1,nx,ny)]
  end

  # Calculating vertex coordinates (nx * ny * nz)
  V = Array{Int64}[]
  for z in 0:nz
    for y in 0:ny
      for x in 0:nx
        push!(V,[x,y,z])
      end
    end
  end


  # Building CV relationship
  CV = Array{Int64}[]
  for z in 0:nz-1
    for y in 0:ny-1
      for x in 0:nx-1
        push!(CV,the3Dcell([x,y,z]))
      end
    end
  end

  # Building FV relationship
  FV = Array{Int64}[]
  v2coords = invertIndex(nx,ny,nz)

  for h in 0:(length(V)-1)
    x,y,z = v2coords(h)

    if (x < nx) && (y < ny)
      push!(FV, [h,ind(x+1,y,z,nx,ny),ind(x,y+1,z,nx,ny),ind(x+1,y+1,z,nx,ny)])
    end

    if (x < nx) && (z < nz)
      push!(FV, [h,ind(x+1,y,z,nx,ny),ind(x,y,z+1,nx,ny),ind(x+1,y,z+1,nx,ny)])
    end

    if (y < ny) && (z < nz)
      push!(FV,[h,ind(x,y+1,z,nx,ny),ind(x,y,z+1,nx,ny),ind(x,y+1,z+1,nx,ny)])
    end

  end

  # Building VV relationship
  VV = map((x)->[x], 0:length(V)-1)

  # Building EV relationship
  EV = Array{Int64}[]
  for h in 0:length(V)-1
    x,y,z = v2coords(h)
    if (x < nx)
      push!(EV, [h,ind(x+1,y,z,nx,ny)])
    end
    if (y < ny)
      push!(EV, [h,ind(x,y+1,z,nx,ny)])
    end
    if (z < nz)
      push!(EV, [h,ind(x,y,z+1,nx,ny)])
    end
  end

  # return all basis
  return V, (VV, EV, FV, CV)
end
end
@}


@O src/model2Obj.jl
@{module Model2Obj
"""
Module that takes a 3d model and write it on
obj files
"""

require(Pkg.dir("ImagesToLARModel/src"), "/larUtils.jl"))

import LARUtils

using Logging

export writeToObj, mergeObj

function writeToObj(imageDx, imageDy, imageDz,
                    xStart, yStart, zStart,
                    objectBoundaryChain, outputFilename)
  """
  Takes the boundary chain of a part of the model
  and writes it on stl files
  """
  V, bases = LARUtils.getBases(imageDx, imageDy, imageDz)
  FV = bases[3]

  outputVtx = string(outputFilename, "_vtx.stl")
  outputFaces = string(outputFilename, "_faces.stl")

  fileVertex = open(outputVtx, "w")
  fileFaces = open(outputFaces, "w")

  vertex_count = 1
  count = 0

  #b2cells = Lar2Julia.cscChainToCellList(objectBoundaryChain)
  # Get all cells (independently from orientation)
  b2cells = findn(objectBoundaryChain)[1]

  debug("b2cells = ", b2cells)

  for f in b2cells
    old_vertex_count = vertex_count
    for vtx in FV[f]
      write(fileVertex, "v ")
      write(fileVertex, string(convert(Int64, V[vtx + 1][1] + xStart)))
      write(fileVertex, " ")
      write(fileVertex, string(convert(Int64, V[vtx + 1][2] + yStart)))
      write(fileVertex, " ")
      write(fileVertex, string(convert(Int64, V[vtx + 1][3] + zStart)))
      write(fileVertex, "\n")
      vertex_count += 1
    end

    write(fileFaces, "f ")
    write(fileFaces, string(old_vertex_count))
    write(fileFaces, " ")
    write(fileFaces, string(old_vertex_count + 1))
    write(fileFaces, " ")
    write(fileFaces, string(old_vertex_count + 3))
    write(fileFaces, "\n")

    write(fileFaces, "f ")
    write(fileFaces, string(old_vertex_count))
    write(fileFaces, " ")
    write(fileFaces, string(old_vertex_count + 3))
    write(fileFaces, " ")
    write(fileFaces, string(old_vertex_count + 2))
    write(fileFaces, "\n")

  end

  close(fileVertex)
  close(fileFaces)

end

function mergeObj(modelDirectory)
  """
  Merge stl files in a single obj file

  modelDirectory: directory containing models
  """

  files = readdir(modelDirectory)
  vertices_files = files[find(s -> contains(s,string("_vtx.stl")), files)]
  faces_files = files[find(s -> contains(s,string("_faces.stl")), files)]
  obj_file = open(string(modelDirectory,"/","model.obj"),"w") # Output file

  vertices_counts = Array(Int64, length(vertices_files))
  number_of_vertices = 0
  for i in 1:length(vertices_files)
    vtx_file = vertices_files[i]
    f = open(string(modelDirectory, "/", vtx_file))

    # Writing vertices on the obj file
    for ln in eachline(f)
      write(obj_file, ln)
      number_of_vertices += 1
    end
    # Saving number of vertices
    vertices_counts[i] = number_of_vertices
    close(f)
  end

  for i in 1 : length(faces_files)
    faces_file = faces_files[i]
    f = open(string(modelDirectory, "/", faces_file))

    for ln in eachline(f)
      splitted = split(ln)
      write(obj_file, "f ")
      if i > 1
        write(obj_file, string(parse(splitted[2]) + vertices_counts[i - 1], " "))
        write(obj_file, string(parse(splitted[3]) + vertices_counts[i - 1], " "))
        write(obj_file, string(parse(splitted[4]) + vertices_counts[i - 1]))
      else
        write(obj_file, string(splitted[2], " "))
        write(obj_file, string(splitted[3], " "))
        write(obj_file, splitted[4])
      end
      write(obj_file, "\n")
    end
    close(f)
  end
  close(obj_file)

  # Removing all tmp files
  for vtx_file in vertices_files
    rm(string(modelDirectory, "/", vtx_file))
  end

  for fcs_file in faces_files
    rm(string(modelDirectory, "/", fcs_file))
  end

end
end
@}


@O src/pngStack2Array3dJulia.jl
@{module PngStack2Array3dJulia

"""
This module loads a stack of png files returning
an array of pixel values divided into segments
"""

export calculateClusterCentroids, pngstack2array3d, getImageData, convertImages

using Images # For loading png images
using Colors # For grayscale images
using PyCall # For including python clustering
using Logging
@@pyimport scipy.ndimage as ndimage
@@pyimport scipy.cluster.vq as cluster

NOISE_SHAPE_DETECT=10

function getImageData(imageFile)
  """
  Get width and heigth from a png image
  """

  input = open(imageFile, "r")
  data = readbytes(input, 24)

  if (data[2:4] != [80, 78, 71] && data[13:16] != [73, 72, 68, 82])
    error("This is not a png image")
  end

  w = data[17:20]
  h = data[21:24]

  width = reinterpret(Int32, reverse(w))[1]
  height = reinterpret(Int32, reverse(h))[1]

  close(input)

  return width, height
end

function calculateClusterCentroids(path, image, numberOfClusters = 2)
  """
  Loads an image and calculate cluster centroids for segmentation

  path: Path of the image folder
  image: name of the image
  numberOfClusters: number of desidered clusters
  """
  imageFilename = string(path, image)

  img = imread(imageFilename) # Open png image with Julia Package

  rgb_img = convert(Image{ColorTypes.RGB}, img)
  gray_img = convert(Image{ColorTypes.Gray}, rgb_img)
  imArray = raw(gray_img)

  imageWidth = size(imArray)[1]
  imageHeight = size(imArray)[2]

  # Getting pixel values and saving them with another shape
  image3d = Array(Array{Uint8,2}, 0)

  # Inserting page on another list and reshaping
  push!(image3d, imArray)
  pixel = reshape(image3d[1], (imageWidth * imageHeight), 1)

  # Segmenting image using kmeans
  # https://en.wikipedia.org/wiki/Image_segmentation#Clustering_methods

  centroids,_ = cluster.kmeans(pixel, numberOfClusters)

  return centroids

end


function pngstack2array3d(path, minSlice, maxSlice, centroids)
  """
  Import a stack of PNG images into a 3d array

  path: path of images directory
  minSlice and maxSlice: number of first and last slice
  centroids: centroids for image segmentation
  """

  # image3d contains all images values
  image3d = Array(Array{Uint8,2}, 0)

  debug("maxSlice = ", maxSlice, " minSlice = ", minSlice)
  files = readdir(path)

  for slice in minSlice : (maxSlice - 1)
    debug("slice = ", slice)
    imageFilename = string(path, files[slice + 1])
    debug("image name: ", imageFilename)
    img = imread(imageFilename) # Open png image with Julia Package

    # Converting image in grayscale
    rgb_img = convert(Image{ColorTypes.RGB}, img)
    gray_img = convert(Image{ColorTypes.Gray}, rgb_img)
    imArray = raw(gray_img) # Putting pixel values into RAW 3d array
    debug("imArray size: ", size(imArray))

    # Inserting page on another list and reshaping
    push!(image3d, imArray)

  end

  # Removing noise using a median filter and quantization
  for page in 1:length(image3d)

    # Denoising
    image3d[page] = ndimage.median_filter(image3d[page], NOISE_SHAPE_DETECT)

    # Image Quantization
    debug("page = ", page)
    debug("image3d[page] dimensions: ", size(image3d[page])[1], "\t", size(image3d[page])[2])
    pixel = reshape(image3d[page], size(image3d[page])[1] * size(image3d[page])[2] , 1)
    qnt,_ = cluster.vq(pixel,centroids)

    # Reshaping quantization result
    centers_idx = reshape(qnt, size(image3d[page],1), size(image3d[page],2))
    #centers_idx = reshape(qnt, size(image3d[page]))

    # Inserting quantized values into 3d image array
    tmp = Array(Uint8, size(image3d[page],1), size(image3d[page],2))

    for j in 1:size(image3d[1],2)
      for i in 1:size(image3d[1],1)
        tmp[i,j] = centroids[centers_idx[i,j] + 1]
      end
    end

    image3d[page] = tmp

  end

  return image3d
end

function convertImages(inputPath, outputPath, bestImage)
  """
  Get all images contained in inputPath directory
  saving them in outputPath directory in png format.
  If images have one of two odd dimensions, they will be resized
  and if folder contains an odd number of images another one will be
  added

  inputPath: Directory containing input images
  outputPath: Temporary directory containing png images
  bestImage: Image chosen for centroids computation

  Returns the new name for the best image
  """

  imageFiles = readdir(inputPath)
  numberOfImages = length(imageFiles)
  outputPrefix = ""
  for i in 1: length(string(numberOfImages)) - 1
    outputPrefix = string(outputPrefix,"0")
  end

  newBestImage = ""
  imageNumber = 0
  for imageFile in imageFiles
    img = imread(string(inputPath, imageFile))

    # resizing images if they do not have even dimensions
    dim = size(img)
    if(dim[1] % 2 != 0)
      debug("Image has odd x; resizing")
      xrange = 1: dim[1] - 1
    else
      xrange = 1: dim[1]
    end

    if(dim[2] % 2 != 0)
      debug("Image has odd y; resizing")
      yrange = 1: dim[2] - 1
    else
      yrange = 1: dim[2]
    end

    img = subim(img, xrange, yrange)

    outputFilename = string(outputPath, outputPrefix[length(string(imageNumber)):end], imageNumber,".png")
    imwrite(img, outputFilename)

    # Searching the best image
    if(imageFile == bestImage)
      newBestImage = string(outputPrefix[length(string(imageNumber)):end], imageNumber,".png")
    end

    imageNumber += 1
  end

  # Adding another image if they are odd
  if(numberOfImages % 2 != 0)
    debug("Odd images, adding one")
    bestImage = imread(string(outputPath, "/", newBestImage))
    imArray = zeros(Uint8, size(bestImage))
    img = grayim(imArray)
    outputFilename = string(outputPath, "/", outputPrefix[length(string(imageNumber)):end], imageNumber,".png")
    imwrite(img, outputFilename)
  end

  return newBestImage
end

end
@}




%===============================================================================
\subsection{Installing the library}
%===============================================================================

%===============================================================================
\section{Conclusions}\label{sec:conclusions}
%===============================================================================
%-------------------------------------------------------------------------------
\subsection{Results}
%-------------------------------------------------------------------------------

%-------------------------------------------------------------------------------
\subsection{Further improvements}
%-------------------------------------------------------------------------------

%-------------------------------------------------------------------------------
\bibliographystyle{amsalpha}
\bibliography{ImagesToLARModel}
%-------------------------------------------------------------------------------
%===============================================================================
\appendix
\section{Utility functions}
%===============================================================================

%-------------------------------------------------------------------------------

\section{Tests}\label{sec:tests}

\paragraph{Generation of the border matrix}
%-------------------------------------------------------------------------------
@O test/generateBorderMatrix.jl
@{push!(LOAD_PATH, "../../")
import GenerateBorderMatrix
import JSON
using Base.Test

function testComputeOriented3Border()
  """
  Test function for computeOriented3Border
  """
  boundaryMatrix = GenerateBorderMatrix.computeOriented3Border(2,2,2)

  rowcount = boundaryMatrix[:shape][1]
  @@test rowcount == 36
  colcount = boundaryMatrix[:shape][2]
  @@test colcount == 8
  row = boundaryMatrix[:indptr]
  @@test row == [0,1,2,3,4,5,7,8,9,11,12,13,15,17,18,19,20,22,23,24,26,27,29,30,32,34,35,37,39,41,42,43,44,45,46,47,48]
  col = boundaryMatrix[:indices]
  @@test col == [0,0,0,1,1,0,1,1,2,0,2,2,3,1,3,2,3,3,2,3,0,4,4,4,1,5,5,4,5,5,2,6,4,6,6,3,7,5,7,6,7,7,6,7,4,5,6,7]
  data = boundaryMatrix[:data]
  @@test data == [-1,1,-1,-1,1,1,-1,1,-1,-1,1,-1,-1,-1,1,1,-1,1,-1,-1,1,-1,1,-1,1,-1,1,1,-1,1,1,-1,-1,1,-1,1,-1,-1,1,1,-1,1,-1,-1,1,1,1,1]

end

function testWriteBorder()
  """
  Test for writeBorder
  """
  boundaryMatrix = GenerateBorderMatrix.computeOriented3Border(2,2,2)
  filename = "borderFile"

  GenerateBorderMatrix.writeBorder(boundaryMatrix, filename)
  @@test isfile(filename)

  # Loading borderMatrix from json file
  borderData = JSON.parsefile(filename)
  row = Array(Int64, length(borderData["ROW"]))
  col = Array(Int64, length(borderData["COL"]))
  data = Array(Int64, length(borderData["DATA"]))

  @@test borderData["ROW"] == [0,1,2,3,4,5,7,8,9,11,12,13,15,17,18,19,20,22,23,24,26,27,29,30,32,34,35,37,39,41,42,43,44,45,46,47,48]
  @@test borderData["COL"] == [0,0,0,1,1,0,1,1,2,0,2,2,3,1,3,2,3,3,2,3,0,4,4,4,1,5,5,4,5,5,2,6,4,6,6,3,7,5,7,6,7,7,6,7,4,5,6,7]
  @@test borderData["DATA"] == [-1,1,-1,-1,1,1,-1,1,-1,-1,1,-1,-1,-1,1,1,-1,1,-1,-1,1,-1,1,-1,1,-1,1,1,-1,1,1,-1,-1,1,-1,1,-1,-1,1,1,-1,1,-1,-1,1,1,1,1]

  rm(filename)

end

function executeAllTests()
  @@time testComputeOriented3Border()
  @@time testWriteBorder()
  println("Tests completed.")
end

executeAllTests()

@}
%-------------------------------------------------------------------------------

\paragraph{Conversion of a png stack to a 3D array}
%-------------------------------------------------------------------------------
@O test/pngStack2Array3dJulia.jl
@{push!(LOAD_PATH, "../../")
import PngStack2Array3dJulia
using Base.Test

function testGetImageData()
  """
  Test function for getImageData
  """

  width, height = PngStack2Array3dJulia.getImageData("images/0.png")

  @@test width == 50
  @@test height == 50

end

function testCalculateClusterCentroids()
  """
  Test function for calculateClusterCentroids
  """
  path = "images/"
  image = 0
  centroids = PngStack2Array3dJulia.calculateClusterCentroids(path, image, 2)

  expected = [0, 253]
  centroids = vec(reshape(centroids, 1, 2))

  @@test sort(centroids) == expected
end

function testPngstack2array3d()
  """
  Test function for pngstack2array3d
  """
  path = "images/"
  minSlice = 0
  maxSlice = 4
  centroids = PngStack2Array3dJulia.calculateClusterCentroids(path, 0, 2)
  image3d = PngStack2Array3dJulia.pngstack2array3d(path, minSlice, maxSlice, centroids)

  @@test size(image3d)[1] == 5
  @@test size(image3d[1])[1] == 50
  @@test size(image3d[1])[2] == 200

end

function executeAllTests()
  @@time testCalculateClusterCentroids()
  @@time testPngstack2array3d()
  @@time testGetImageData()
  println("Tests completed.")
end

executeAllTests()

@}
%-------------------------------------------------------------------------------

\paragraph{Test for LAR utilities}
%-------------------------------------------------------------------------------
@O test/LARUtils.jl
@{push!(LOAD_PATH, "../../")
import LARUtils
using Base.Test

function testInd()
  """
  Test function for ind
  """

  nx = 2
  ny = 2

  @@test LARUtils.ind(0, 0, 0, nx, ny) == 0
  @@test LARUtils.ind(1, 1, 1, nx, ny) == 13
  @@test LARUtils.ind(2, 5, 4, nx, ny) == 53
  @@test LARUtils.ind(1, 1, 1, nx, ny) == 13
  @@test LARUtils.ind(2, 7, 1, nx, ny) == 32
  @@test LARUtils.ind(1, 0, 3, nx, ny) == 28
end

function executeAllTests()
  @@time testInd()
  println("Tests completed.")
end

executeAllTests()

@}
%-------------------------------------------------------------------------------


\end{document}
