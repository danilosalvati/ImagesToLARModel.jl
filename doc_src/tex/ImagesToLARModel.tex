\documentclass[11pt,oneside]{article}	%use"amsart"insteadof"article"forAMSLaTeXformat
\usepackage{geometry}		%Seegeometry.pdftolearnthelayoutoptions.Therearelots.
\geometry{letterpaper}		%...ora4paperora5paperor...
%\geometry{landscape}		%Activateforforrotatedpagegeometry
%\usepackage[parfill]{parskip}		%Activatetobeginparagraphswithanemptylineratherthananindent
\usepackage{graphicx}				%Usepdf,png,jpg,orepswithpdflatex;useepsinDVImode
								%TeXwillautomaticallyconverteps-->pdfinpdflatex		
\usepackage{amssymb}
\usepackage[colorlinks]{hyperref}

\usepackage{array}
\newcolumntype{P}[1]{>{\raggedright\arraybackslash}p{#1}}

%----macros begin---------------------------------------------------------------
\usepackage{color}
\usepackage{amsthm}

\def\conv{\mbox{\textrm{conv}\,}}
\def\aff{\mbox{\textrm{aff}\,}}
\def\E{\mathbb{E}}
\def\R{\mathbb{R}}
\def\Z{\mathbb{Z}}
\def\tex{\TeX}
\def\latex{\LaTeX}
\def\v#1{{\bf #1}}
\def\p#1{{\bf #1}}
\def\T#1{{\bf #1}}

\def\vet#1{{\left(\begin{array}{cccccccccccccccccccc}#1\end{array}\right)}}
\def\mat#1{{\left(\begin{array}{cccccccccccccccccccc}#1\end{array}\right)}}

\def\lin{\mbox{\rm lin}\,}
\def\aff{\mbox{\rm aff}\,}
\def\pos{\mbox{\rm pos}\,}
\def\cone{\mbox{\rm cone}\,}
\def\conv{\mbox{\rm conv}\,}
\newcommand{\homog}[0]{\mbox{\rm homog}\,}
\newcommand{\relint}[0]{\mbox{\rm relint}\,}

%----macros end-----------------------------------------------------------------

\title{ImagesToLARModel, a tool for creation of three-dimensional models from a stack of images}
\author{Danilo Salvati}
%\date{}							%Activatetodisplayagivendateornodate

\begin{document}
\maketitle
%\nonstopmode

\begin{abstract}
This is the abstract (we will use LAR~\cite{cclar-proj:2013:00})

\end{abstract}

\newpage
\tableofcontents
\newpage

%-------------------------------------------------------------------------------
%===============================================================================
\section{Introduction}\label{sec:intro}
%===============================================================================

%-------------------------------------------------------------------------------
%-------------------------------------------------------------------------------
\section{Exporting the library}
%-------------------------------------------------------------------------------

@O src/ImagesToLARModel.jl
@{module ImagesToLARModel
"""
Main module for the library. It starts conversion
taking configuration parameters
"""
require(string(Pkg.dir("ImagesToLARModel/src"), "/imagesConvertion.jl"))

import JSON
import ImagesConvertion

using Logging

export convertImagesToLARModel

function loadConfiguration(configurationFile)
  """
  load parameters from JSON file
  
  configurationFile: Path of the configuration file
  """

  configuration = JSON.parse(configurationFile)
  
  DEBUG_LEVELS = [DEBUG, INFO, WARNING, ERROR, CRITICAL]
  
  try
    if configuration["parallelMerge"] == "true"
      parallelMerge = true
    else
      parallelMerge = false
    end
  catch
    parallelMerge = false
  end

  return configuration["inputDirectory"], configuration["outputDirectory"], configuration["bestImage"],
        configuration["nx"], configuration["ny"], configuration["nz"],
        DEBUG_LEVELS[configuration["DEBUG_LEVEL"]]

end

function convertImagesToLARModel(configurationFile)
  """
  Start convertion of a stack of images into a 3D model
  loading parameters from a JSON configuration file
  
  configurationFile: Path of the configuration file
  """
  inputDirectory, outputDirectory, bestImage, nx, ny, nz, DEBUG_LEVEL = loadConfiguration(open(configurationFile))
  convertImagesToLARModel(inputDirectory, outputDirectory, bestImage, nx, ny, nz, DEBUG_LEVEL)
end

function convertImagesToLARModel(inputDirectory, outputDirectory, bestImage,
                                 nx, ny, nz, DEBUG_LEVEL = INFO, parallelMerge = false)
  """
  Start convertion of a stack of images into a 3D model
  
  inputDirectory: Directory containing the stack of images
  outputDirectory: Directory containing the output
  bestImage: Image chosen for centroids computation
  nx, ny, nz: Border dimensions (Possibly the biggest power of two of images dimensions)
  DEBUG_LEVEL: Debug level for Julia logger. It can be one of the following:
    - DEBUG
    - INFO
    - WARNING
    - ERROR
    - CRITICAL
  """
  # Create output directory
  try
    mkpath(outputDirectory)
  catch
  end

  Logging.configure(level=DEBUG_LEVEL)
  ImagesConvertion.images2LARModel(nx, ny, nz, bestImage, inputDirectory, outputDirectory, parallelMerge)
end
end
@}

@O src/imagesConvertion.jl
@{module ImagesConvertion

require(string(Pkg.dir("ImagesToLARModel/src"), "/generateBorderMatrix.jl"))
require(string(Pkg.dir("ImagesToLARModel/src"), "/pngStack2Array3dJulia.jl"))
require(string(Pkg.dir("ImagesToLARModel/src"), "/lar2Julia.jl"))
require(string(Pkg.dir("ImagesToLARModel/src"), "/model2Obj.jl"))
require(string(Pkg.dir("ImagesToLARModel/src"), "/larUtils.jl"))

import GenerateBorderMatrix
import PngStack2Array3dJulia
import Lar2Julia
import Model2Obj
import LARUtils

import JSON

using PyCall
@@pyimport scipy.sparse as Pysparse

using Logging

export images2LARModel

"""
This is main module for converting a stack
of images into a 3d model
"""

function images2LARModel(nx, ny, nz, bestImage, inputDirectory, outputDirectory, parallelMerge)
  """
  Convert a stack of images into a 3d model
  """

  info("Starting model creation")

  numberOfClusters = 2 # Number of clusters for
                       # images segmentation

  info("Moving images into temp directory")
  try
    mkdir(string(outputDirectory, "TEMP"))
  catch
  end

  tempDirectory = string(outputDirectory,"TEMP/")

  newBestImage = PngStack2Array3dJulia.convertImages(inputDirectory, tempDirectory, bestImage)

  imageWidth, imageHeight = PngStack2Array3dJulia.getImageData(string(tempDirectory,newBestImage))
  imageDepth = length(readdir(tempDirectory))

  # Computing border matrix
  info("Computing border matrix")
  try
    mkdir(string(outputDirectory, "BORDERS"))
  catch
  end
  borderFilename = GenerateBorderMatrix.getOriented3BorderPath(string(outputDirectory, "BORDERS"), nx, ny, nz)

  # Starting images convertion and border computation
  info("Starting images convertion")
  startImageConvertion(tempDirectory, newBestImage, outputDirectory, borderFilename,
                       imageHeight, imageWidth, imageDepth,
                       nx, ny, nz,
                       numberOfClusters, parallelMerge)

end

function startImageConvertion(sliceDirectory, bestImage, outputDirectory, borderFilename,
                              imageHeight, imageWidth, imageDepth,
                              imageDx, imageDy, imageDz,
                              numberOfClusters, parallelMerge)
  """
  Support function for converting a stack of images into a model

  sliceDirectory: directory containing the image stack
  imageForCentroids: image chosen for centroid computation
  """

  # Create clusters for image segmentation
  info("Computing image centroids")
  debug("Best image = ", bestImage)
  centroidsCalc = PngStack2Array3dJulia.calculateClusterCentroids(sliceDirectory, bestImage, numberOfClusters)
  debug(string("centroids = ", centroidsCalc))

  try
    mkdir(string(outputDirectory, "BORDERS"))
  catch
  end
  debug(string("Opening border file: ", "border_", imageDx, "-", imageDy, "-", imageDz, ".json"))
  boundaryMat = getBorderMatrix(string(outputDirectory,"BORDERS/","border_", imageDx, "-",
                                       imageDy, "-", imageDz, ".json"))
  beginImageStack = 0
  endImage = beginImageStack

  info("Converting images into a 3d model")
  tasks = Array(RemoteRef, 0)
  for zBlock in 0:(imageDepth / imageDz - 1)
    startImage = endImage
    endImage = startImage + imageDz
    info("StartImage = ", startImage)
    info("endImage = ", endImage)

    #=
    task = @@spawn imageConvertionProcess(sliceDirectory, outputDirectory,
                           beginImageStack, startImage, endImage,
                           imageDx, imageDy, imageDz,
                           imageHeight, imageWidth,
                           centroidsCalc, boundaryMat)

    push!(tasks, task)
    =#
    imageConvertionProcess(sliceDirectory, outputDirectory,
                           beginImageStack, startImage, endImage,
                           imageDx, imageDy, imageDz,
                           imageHeight, imageWidth,
                           centroidsCalc, boundaryMat)

  end

  # Waiting for tasks completion
  for task in tasks
    wait(task)
  end

  info("Merging boundaries")
  # Merge Boundaries files
  Model2Obj.mergeBoundaries(string(outputDirectory, "MODELS"),
                            imageHeight, imageWidth, imageDepth,
                            imageDx, imageDy, imageDz)

  info("Merging obj models")
  if parallelMerge
    Model2Obj.mergeObjParallel(string(outputDirectory, "MODELS"))
  else
    Model2Obj.mergeObj(string(outputDirectory, "MODELS"))
  end

end

function imageConvertionProcess(sliceDirectory, outputDirectory,
                                beginImageStack, startImage, endImage,
                                imageDx, imageDy, imageDz,
                                imageHeight, imageWidth,
                                centroids, boundaryMat)
  """
  Support function for converting a stack of image on a single
  independent process
  """

  info("Transforming png data into 3d array")
  theImage = PngStack2Array3dJulia.pngstack2array3d(sliceDirectory, startImage, endImage, centroids)

  centroidsSorted = sort(vec(reshape(centroids, 1, 2)))
  foreground = centroidsSorted[2]
  background = centroidsSorted[1]
  debug(string("background = ", background, " foreground = ", foreground))

  for xBlock in 0:(imageHeight / imageDx - 1)
    for yBlock in 0:(imageWidth / imageDy - 1)
      yStart = xBlock * imageDx
      xStart = yBlock * imageDy
      #xEnd = xStart + imageDx
      #yEnd = yStart + imageDy
      xEnd = xStart + imageDy
      yEnd = yStart + imageDx
      debug("***********")
      debug(string("xStart = ", xStart, " xEnd = ", xEnd))
      debug(string("yStart = ", yStart, " yEnd = ", yEnd))
      debug("theImage dimensions: ", size(theImage)[1], " ", size(theImage[1])[1], " ", size(theImage[1])[2])

      # Getting a slice of theImage array

      image = Array(Uint8, (convert(Int, length(theImage)), convert(Int, xEnd - xStart), convert(Int, yEnd - yStart)))
      debug("image size: ", size(image))
      for z in 1:length(theImage)
        for x in 1 : (xEnd - xStart)
          for y in 1 : (yEnd - yStart)
            image[z, x, y] = theImage[z][x + xStart, y + yStart]
          end
        end
      end

      nx, ny, nz = size(image)
      chains3D = Array(Uint8, 0)
      zStart = startImage - beginImageStack
      for y in 0:(nx - 1)
        for x in 0:(ny - 1)
          for z in 0:(nz - 1)
            if(image[z + 1, x + 1, y + 1] == foreground)
              push!(chains3D, y + ny * (x + nx * z))
            end
          end
        end
      end

      if(length(chains3D) != 0)
        # Computing boundary chain
        debug("chains3d = ", chains3D)
        debug("Computing boundary chain")
        objectBoundaryChain = Lar2Julia.larBoundaryChain(boundaryMat, chains3D)
        debug("Converting models into obj")
        try
          mkdir(string(outputDirectory, "MODELS"))
        catch
        end
        # IMPORTANT: inverting xStart and yStart for obtaining correct rotation of the model
        models = LARUtils.computeModelAndBoundaries(imageDx, imageDy, imageDz, yStart, xStart, zStart, objectBoundaryChain)

        V, FV = models[1][1] # inside model
        V_left, FV_left = models[2][1]
        V_right, FV_right = models[3][1] # right boundary
        V_top, FV_top = models[4][1] # top boundary
        V_bottom, FV_bottom = models[5][1] # bottom boundary
        V_front, FV_front = models[6][1] # front boundary
        V_back, FV_back = models[7][1] # back boundary

        # Writing all models on disk
        model_outputFilename = string(outputDirectory, "MODELS/model_output_", xBlock, "-", yBlock, "_", startImage, "_", endImage)
        Model2Obj.writeToObj(V, FV, model_outputFilename)

        left_outputFilename = string(outputDirectory, "MODELS/left_output_", xBlock, "-", yBlock, "_", startImage, "_", endImage)
        Model2Obj.writeToObj(V_left, FV_left, left_outputFilename)

        right_outputFilename = string(outputDirectory, "MODELS/right_output_", xBlock, "-", yBlock, "_", startImage, "_", endImage)
        Model2Obj.writeToObj(V_right, FV_right, right_outputFilename)

        top_outputFilename = string(outputDirectory, "MODELS/top_output_", xBlock, "-", yBlock, "_", startImage, "_", endImage)
        Model2Obj.writeToObj(V_top, FV_top, top_outputFilename)

        bottom_outputFilename = string(outputDirectory, "MODELS/bottom_output_", xBlock, "-", yBlock, "_", startImage, "_", endImage)
        Model2Obj.writeToObj(V_bottom, FV_bottom, bottom_outputFilename)

        front_outputFilename = string(outputDirectory, "MODELS/front_output_", xBlock, "-", yBlock, "_", startImage, "_", endImage)
        Model2Obj.writeToObj(V_front, FV_front, front_outputFilename)

        back_outputFilename = string(outputDirectory, "MODELS/back_output_", xBlock, "-", yBlock, "_", startImage, "_", endImage)
        Model2Obj.writeToObj(V_back, FV_back, back_outputFilename)
      else
        debug("Model is empty")
      end
    end
  end
end

function getBorderMatrix(borderFilename)
  """
  TO REMOVE WHEN PORTING OF LARCC IN JULIA IS COMPLETED

  Get the border matrix from json file and convert it in
  CSC format
  """
  # Loading borderMatrix from json file
  borderData = JSON.parsefile(borderFilename)
  row = Array(Int64, length(borderData["ROW"]))
  col = Array(Int64, length(borderData["COL"]))
  data = Array(Int64, length(borderData["DATA"]))

  for i in 1: length(borderData["ROW"])
    row[i] = borderData["ROW"][i]
  end

  for i in 1: length(borderData["COL"])
    col[i] = borderData["COL"][i]
  end

  for i in 1: length(borderData["DATA"])
    data[i] = borderData["DATA"][i]
  end

  # Converting csr matrix to csc
  csrBorderMatrix = Pysparse.csr_matrix((data,col,row), shape=(borderData["ROWCOUNT"],borderData["COLCOUNT"]))
  denseMatrix = pycall(csrBorderMatrix["toarray"],PyAny)

  cscBoundaryMat = sparse(denseMatrix)

  return cscBoundaryMat

end
end
@}


@O src/generateBorderMatrix.jl
@{module GenerateBorderMatrix
"""
Module for generation of the boundary matrix
"""

type MatrixObject
  ROWCOUNT
  COLCOUNT
  ROW
  COL
  DATA
end


export computeOriented3Border, writeBorder, getOriented3BorderPath

require(string(Pkg.dir("ImagesToLARModel/src"), "/larUtils.jl"))

import LARUtils
using PyCall

import JSON

@@pyimport sys
unshift!(PyVector(pyimport("sys")["path"]), "") # Search for python modules in folder
# Search for python modules in package folder
unshift!(PyVector(pyimport("sys")["path"]), Pkg.dir("ImagesToLARModel/src"))
@@pyimport larcc # Importing larcc from local folder

# Compute the 3-border operator
function computeOriented3Border(nx, ny, nz)
  """
  Compute the 3-border matrix using a modified
  version of larcc
  """
  V, bases = LARUtils.getBases(nx, ny, nz)
  boundaryMat = larcc.signedCellularBoundary(V, bases)
  return boundaryMat

end

function writeBorder(boundaryMatrix, outputFile)
  """
  Write 3-border matrix on json file

  boundaryMatrix: matrix to write on file
  outputFile: path of the outputFile
  """

  rowcount = boundaryMatrix[:shape][1]
  colcount = boundaryMatrix[:shape][2]

  row = boundaryMatrix[:indptr]
  col = boundaryMatrix[:indices]
  data = boundaryMatrix[:data]

  # Writing informations on file
  outfile = open(outputFile, "w")

  matrixObj = MatrixObject(rowcount, colcount, row, col, data)
  JSON.print(outfile, matrixObj)
  close(outfile)

end

function getOriented3BorderPath(borderPath, nx, ny, nz)
  """
  Try reading 3-border matrix from file. If it fails matrix
  is computed and saved on disk in JSON format

  borderPath: path of border directory
  nx, ny, nz: image dimensions
  """

  filename = string(borderPath,"/border_", nx, "-", ny, "-", nz, ".json")
  if !isfile(filename)
    border = computeOriented3Border(nx, ny, nz)
    writeBorder(border, filename)
  end
  return filename

end
end
@}


@O src/lar2Julia.jl
@{module Lar2Julia
"""
larcc functions for Julia
"""
export larBoundaryChain, cscChainToCellList

import JSON

using Logging

function larBoundaryChain(cscBoundaryMat, brcCellList)
  """
  Compute boundary chains
  """

  # Computing boundary chains
  n = size(cscBoundaryMat)[1]
  m = size(cscBoundaryMat)[2]

  debug("Boundary matrix size: ", n, "\t", m)

  data = ones(Int64, length(brcCellList))

  i = Array(Int64, length(brcCellList))
  for k in 1:length(brcCellList)
    i[k] = brcCellList[k] + 1
  end

  j = ones(Int64, length(brcCellList))

  debug("cscChain rows length: ", length(i))
  debug("cscChain columns length: ", length(j))
  debug("cscChain data length: ", length(brcCellList))

  debug("rows ", i)
  debug("columns ", j)
  debug("data ", data)

  cscChain = sparse(i, j, data, m, 1)
  cscmat = cscBoundaryMat * cscChain
  out = cscBinFilter(cscmat)
  return out
end

function cscBinFilter(CSCm)
  k = 1
  data = nonzeros(CSCm)
  sgArray = copysign(1, data)

  while k <= nnz(CSCm)
    if data[k] % 2 == 1 || data[k] % 2 == -1
      data[k] = 1 * sgArray[k]
    else
      data[k] = 0
    end
    k += 1
  end

  return CSCm
end

function cscChainToCellList(CSCm)
  """
  Get a csc containing a chain and returns
  the cell list of the "+1" oriented faces
  """
  data = nonzeros(CSCm)
  # Now I need to remove zero element (problem with Julia nonzeros)
  nonzeroData = Array(Int64, 0)
  for n in data
    if n != 0
      push!(nonzeroData, n)
    end
  end

  cellList = Array(Int64,0)
  for (k, theRow) in enumerate(findn(CSCm)[1])
    if nonzeroData[k] == 1
      push!(cellList, theRow)
    end
  end
  return cellList
end
end
@}

@O src/larUtils.jl
@{module LARUtils
"""
Utility functions for extracting 3d models from images
"""

using Logging

export ind, invertIndex, getBases, removeDoubleVerticesAndFaces, computeModel, computeModelAndBoundaries

function ind(x, y, z, nx, ny)
    """
    Transform coordinates into linearized matrix indexes
    """
    return x + (nx + 1) * (y + (ny + 1) * (z))
  end


function invertIndex(nx,ny,nz)
  """
  Invert indexes
  """
  nx, ny, nz = nx + 1, ny + 1, nz + 1
  function invertIndex0(offset)
      a0, b0 = trunc(offset / nx), offset % nx
      a1, b1 = trunc(a0 / ny), a0 % ny
      a2, b2 = trunc(a1 / nz), a1 % nz
      return b0, b1, b2
  end
  return invertIndex0
end


function getBases(nx, ny, nz)
  """
  Compute all LAR relations
  """

  function the3Dcell(coords)
    x,y,z = coords
    return [ind(x,y,z,nx,ny),ind(x+1,y,z,nx,ny),ind(x,y+1,z,nx,ny),ind(x,y,z+1,nx,ny),ind(x+1,y+1,z,nx,ny),
            ind(x+1,y,z+1,nx,ny),ind(x,y+1,z+1,nx,ny),ind(x+1,y+1,z+1,nx,ny)]
  end

  # Calculating vertex coordinates (nx * ny * nz)
  V = Array{Int64}[]
  for z in 0:nz
    for y in 0:ny
      for x in 0:nx
        push!(V,[x,y,z])
      end
    end
  end

  # Building CV relationship
  CV = Array{Int64}[]
  for z in 0:nz-1
    for y in 0:ny-1
      for x in 0:nx-1
        push!(CV,the3Dcell([x,y,z]))
      end
    end
  end

  # Building FV relationship
  FV = Array{Int64}[]
  v2coords = invertIndex(nx,ny,nz)

  for h in 0:(length(V)-1)
    x,y,z = v2coords(h)

    if (x < nx) && (y < ny)
      push!(FV, [h,ind(x+1,y,z,nx,ny),ind(x,y+1,z,nx,ny),ind(x+1,y+1,z,nx,ny)])
    end

    if (x < nx) && (z < nz)
      push!(FV, [h,ind(x+1,y,z,nx,ny),ind(x,y,z+1,nx,ny),ind(x+1,y,z+1,nx,ny)])
    end

    if (y < ny) && (z < nz)
      push!(FV,[h,ind(x,y+1,z,nx,ny),ind(x,y,z+1,nx,ny),ind(x,y+1,z+1,nx,ny)])
    end

  end

  # Building VV relationship
  VV = map((x)->[x], 0:length(V)-1)

  # Building EV relationship
  EV = Array{Int64}[]
  for h in 0:length(V)-1
    x,y,z = v2coords(h)
    if (x < nx)
      push!(EV, [h,ind(x+1,y,z,nx,ny)])
    end
    if (y < ny)
      push!(EV, [h,ind(x,y+1,z,nx,ny)])
    end
    if (z < nz)
      push!(EV, [h,ind(x,y,z+1,nx,ny)])
    end
  end

  # return all basis
  return V, (VV, EV, FV, CV)
end

function lessThanVertices(v1, v2)
  """
  Utility function for comparing vertices coordinates
  """

  if v1[1] == v2[1]
    if v1[2] == v2[2]
      return v1[3] < v2[3]
    end
    return v1[2] < v2[2]
  end
  return v1[1] < v2[1]
end

function removeDoubleVerticesAndFaces(V, FV, facesOffset)
  """
  Removes double vertices and faces from a LAR model

  V: Array containing all vertices
  FV: Array containing all faces
  facesOffset: offset for faces indices
  """

  newV, indices = removeDoubleVertices(V)
  reindexedFaces = reindexVerticesInFaces(FV, indices, facesOffset)
  newFV = unique(FV)

  return newV, newFV

end

function removeDoubleVertices(V)
  """
  Remove double vertices from a LAR model

  V: Array containing all vertices of the model
  """

  # Sort the vertices list and returns the ordered indices
  orderedIndices = sortperm(V, lt = lessThanVertices, alg=MergeSort)

  orderedVerticesAndIndices = collect(zip(sort(V, lt = lessThanVertices),
                                          orderedIndices))
  newVertices = Array(Array{Int}, 0)
  indices = zeros(Int, length(V))
  prevv = Nothing
  i = 1
  for (v, ind) in orderedVerticesAndIndices
    if v == prevv
      indices[ind] = i - 1
    else
      push!(newVertices, v)
      indices[ind] = i
      i += 1
      prevv = v
    end
  end
  return newVertices, indices
end

function reindexVerticesInFaces(FV, indices, offset)
  """
  Reindex vertices indices in faces array

  FV: Faces array of the LAR model
  indices: new Indices for faces
  offset: offset for faces indices
  """

  for f in FV
    for i in 1: length(f)
      f[i] = indices[f[i] - offset] + offset
    end
  end
  return FV
end

function removeVerticesAndFacesFromBoundaries(V, FV)
  """
  Remove vertices and faces duplicates on
  boundaries models

  V,FV: lar model of two merged boundaries
  """

  newV, indices = removeDoubleVertices(V)
  uniqueIndices = unique(indices)

  # Removing double faces on both boundaries
  FV_reindexed = reindexVerticesInFaces(FV, indices, 0)
  FV_unique = unique(FV_reindexed)
  
  FV_cleaned = Array(Array{Int}, 0)
  for f in FV_unique
    if(count((x) -> x == f, FV_reindexed) == 1)
      push!(FV_cleaned, f)
    end
  end
  
  # Creating an array of faces with explicit vertices
  FV_vertices = Array(Array{Array{Int}}, 0)
  
  for i in 1 : length(FV_cleaned)
    push!(FV_vertices, Array(Array{Int}, 0))
    for vtx in FV_cleaned[i]
      push!(FV_vertices[i], newV[vtx])
    end
  end
  
  V_final = Array(Array{Int}, 0)
  FV_final = Array(Array{Int}, 0)
  
  # Saving only used vertices
  for face in FV_vertices
    for vtx in face
      push!(V_final, vtx)
    end
  end
  
  V_final = unique(V_final)

  # Renumbering FV
  for face in FV_vertices
    tmp = Array(Int, 0)
    for vtx in face
      ind = findfirst(V_final, vtx)
      push!(tmp, ind)
    end
    push!(FV_final, tmp)
  end

  return V_final, FV_final
end

function computeModel(imageDx, imageDy, imageDz,
                      xStart, yStart, zStart,
                      facesOffset, objectBoundaryChain)
  """
  Takes the boundary chain of a part of the entire model
  and returns a LAR model

  imageDx, imageDy, imageDz: Boundary dimensions
  xStart, yStart, zStart: Offset of this part of the model
  facesOffset: Offset for the faces
  objectBoundaryChain: Sparse csc matrix containing the cells
  """

  V, bases = getBases(imageDx, imageDy, imageDz)
  FV = bases[3]

  V_model = Array(Array{Int}, 0)
  FV_model = Array(Array{Int}, 0)

  vertex_count = 1

  #b2cells = Lar2Julia.cscChainToCellList(objectBoundaryChain)
  # Get all cells (independently from orientation)
  b2cells = findn(objectBoundaryChain)[1]

  debug("b2cells = ", b2cells)

  for f in b2cells
    old_vertex_count = vertex_count
    for vtx in FV[f]
      push!(V_model, [convert(Int, V[vtx + 1][1] + xStart),
                    convert(Int, V[vtx + 1][2] + yStart),
                    convert(Int, V[vtx + 1][3] + zStart)])
      vertex_count += 1
    end

    push!(FV_model, [old_vertex_count + facesOffset, old_vertex_count + 1 + facesOffset, old_vertex_count + 3 + facesOffset])
    push!(FV_model, [old_vertex_count + facesOffset, old_vertex_count + 3 + facesOffset, old_vertex_count + 2 + facesOffset])
  end

  # Removing double vertices
  return removeDoubleVerticesAndFaces(V_model, FV_model, facesOffset)

end

function isOnLeft(face, V, nx, ny, nz)
  """
  Check if face is on left boundary
  """

  for(vtx in face)
    if(V[vtx + 1][2] != 0)
      return false
    end
  end
  return true

end

function isOnRight(face, V, nx, ny, nz)
  """
  Check if face is on right boundary
  """

  for(vtx in face)
    if(V[vtx + 1][2] != ny)
      return false
    end
  end
  return true

end

function isOnTop(face, V, nx, ny, nz)
  """
  Check if face is on top boundary
  """

  for(vtx in face)
    if(V[vtx + 1][3] != nz)
      return false
    end
  end
  return true
end

function isOnBottom(face, V, nx, ny, nz)
  """
  Check if face is on bottom boundary
  """

  for(vtx in face)
    if(V[vtx + 1][3] != 0)
      return false
    end
  end
  return true
end

function isOnFront(face, V, nx, ny, nz)
  """
  Check if face is on front boundary
  """

  for(vtx in face)
    if(V[vtx + 1][1] != nx)
      return false
    end
  end
  return true
end

function isOnBack(face, V, nx, ny, nz)
  """
  Check if face is on back boundary
  """

  for(vtx in face)
    if(V[vtx + 1][1] != 0)
      return false
    end
  end
  return true
end

function computeModelAndBoundaries(imageDx, imageDy, imageDz,
                      xStart, yStart, zStart,
                      objectBoundaryChain)
  """
  Takes the boundary chain of a part of the entire model
  and returns a LAR model splitting the boundaries

  imageDx, imageDy, imageDz: Boundary dimensions
  xStart, yStart, zStart: Offset of this part of the model
  objectBoundaryChain: Sparse csc matrix containing the cells
  """

  function addFaceToModel(V_base, FV_base, V, FV, face, vertex_count)
    """
    Insert a face into a LAR model

    V_base, FV_base: LAR model of the base
    V, FV: LAR model
    face: Face that will be added to the model
    vertex_count: Indices for faces vertices
    """
    new_vertex_count = vertex_count
    for vtx in FV_base[face]
      push!(V, [convert(Int, V_base[vtx + 1][1] + xStart),
                      convert(Int, V_base[vtx + 1][2] + yStart),
                      convert(Int, V_base[vtx + 1][3] + zStart)])
      new_vertex_count += 1
    end
    push!(FV, [vertex_count, vertex_count + 1, vertex_count + 3])
    push!(FV, [vertex_count, vertex_count + 3, vertex_count + 2])

    return new_vertex_count
  end

  V, bases = getBases(imageDx, imageDy, imageDz)
  FV = bases[3]

  V_model = Array(Array{Int}, 0)
  FV_model = Array(Array{Int}, 0)

  V_left = Array(Array{Int},0)
  FV_left = Array(Array{Int},0)

  V_right = Array(Array{Int},0)
  FV_right = Array(Array{Int},0)

  V_top = Array(Array{Int},0)
  FV_top = Array(Array{Int},0)

  V_bottom = Array(Array{Int},0)
  FV_bottom = Array(Array{Int},0)

  V_front = Array(Array{Int},0)
  FV_front = Array(Array{Int},0)

  V_back = Array(Array{Int},0)
  FV_back = Array(Array{Int},0)

  vertex_count_model = 1
  vertex_count_left = 1
  vertex_count_right = 1
  vertex_count_top = 1
  vertex_count_bottom = 1
  vertex_count_front = 1
  vertex_count_back = 1

  #b2cells = Lar2Julia.cscChainToCellList(objectBoundaryChain)
  # Get all cells (independently from orientation)
  b2cells = findn(objectBoundaryChain)[1]

  debug("b2cells = ", b2cells)

  for f in b2cells
    old_vertex_count_model = vertex_count_model
    old_vertex_count_left = vertex_count_left
    old_vertex_count_right = vertex_count_right
    old_vertex_count_top = vertex_count_top
    old_vertex_count_bottom = vertex_count_bottom
    old_vertex_count_front = vertex_count_front
    old_vertex_count_back = vertex_count_back

    # Choosing the right model for vertex
    if(isOnLeft(FV[f], V, imageDx, imageDy, imageDz))
      vertex_count_left = addFaceToModel(V, FV, V_left, FV_left, f, old_vertex_count_left)
    elseif(isOnRight(FV[f], V, imageDx, imageDy, imageDz))
      vertex_count_right = addFaceToModel(V, FV, V_right, FV_right, f, old_vertex_count_right)
    elseif(isOnTop(FV[f], V, imageDx, imageDy, imageDz))
      vertex_count_top = addFaceToModel(V, FV, V_top, FV_top, f, old_vertex_count_top)
    elseif(isOnBottom(FV[f], V, imageDx, imageDy, imageDz))
      vertex_count_bottom = addFaceToModel(V, FV, V_bottom, FV_bottom, f, old_vertex_count_bottom)
    elseif(isOnFront(FV[f], V, imageDx, imageDy, imageDz))
      vertex_count_front = addFaceToModel(V, FV, V_front, FV_front, f, old_vertex_count_front)
    elseif(isOnBack(FV[f], V, imageDx, imageDy, imageDz))
      vertex_count_back = addFaceToModel(V, FV, V_back, FV_back, f, old_vertex_count_back)
    else
      vertex_count_model = addFaceToModel(V, FV, V_model, FV_model, f, old_vertex_count_model)
    end

  end

  # Removing double vertices
  return [removeDoubleVerticesAndFaces(V_model, FV_model, 0)],
  [removeDoubleVerticesAndFaces(V_left, FV_left, 0)],
  [removeDoubleVerticesAndFaces(V_right, FV_right, 0)],
  [removeDoubleVerticesAndFaces(V_top, FV_top, 0)],
  [removeDoubleVerticesAndFaces(V_bottom, FV_bottom, 0)],
  [removeDoubleVerticesAndFaces(V_front, FV_front, 0)],
  [removeDoubleVerticesAndFaces(V_back, FV_back, 0)]
end
end
@}


@O src/model2Obj.jl
@{module Model2Obj
"""
Module that takes a 3d model and write it on
obj files
"""

require(string(Pkg.dir("ImagesToLARModel/src"), "/larUtils.jl"))

import LARUtils

using Logging

export writeToObj, mergeObj, mergeObjParallel


function writeToObj(V, FV, outputFilename)
  """
  Take a LAR model and write it on obj file

  V: array containing vertices coordinates
  FV: array containing faces
  outputFilename: prefix for the output files
  """

  if (length(V) != 0)
    outputVtx = string(outputFilename, "_vtx.stl")
    outputFaces = string(outputFilename, "_faces.stl")

    fileVertex = open(outputVtx, "w")
    fileFaces = open(outputFaces, "w")

    for v in V
      write(fileVertex, "v ")
      write(fileVertex, string(v[1], " "))
      write(fileVertex, string(v[2], " "))
      write(fileVertex, string(v[3], "\n"))
    end

    for f in FV

      write(fileFaces, "f ")
      write(fileFaces, string(f[1], " "))
      write(fileFaces, string(f[2], " "))
      write(fileFaces, string(f[3], "\n"))
    end

    close(fileVertex)
    close(fileFaces)

  end

end

function mergeObj(modelDirectory)
  """
  Merge stl files in a single obj file

  modelDirectory: directory containing models
  """

  files = readdir(modelDirectory)
  vertices_files = files[find(s -> contains(s,string("_vtx.stl")), files)]
  faces_files = files[find(s -> contains(s,string("_faces.stl")), files)]
  obj_file = open(string(modelDirectory,"/","model.obj"),"w") # Output file

  vertices_counts = Array(Int64, length(vertices_files))
  number_of_vertices = 0
  for i in 1:length(vertices_files)
    vtx_file = vertices_files[i]
    f = open(string(modelDirectory, "/", vtx_file))
    debug("Opening ", vtx_file)

    # Writing vertices on the obj file
    for ln in eachline(f)
      write(obj_file, ln)
      number_of_vertices += 1
    end
    # Saving number of vertices
    vertices_counts[i] = number_of_vertices
    close(f)
  end

  for i in 1 : length(faces_files)
    faces_file = faces_files[i]
    f = open(string(modelDirectory, "/", faces_file))
    debug("Opening ", faces_file)
    for ln in eachline(f)
      splitted = split(ln)
      write(obj_file, "f ")
      if i > 1
        write(obj_file, string(parse(splitted[2]) + vertices_counts[i - 1], " "))
        write(obj_file, string(parse(splitted[3]) + vertices_counts[i - 1], " "))
        write(obj_file, string(parse(splitted[4]) + vertices_counts[i - 1]))
      else
        write(obj_file, string(splitted[2], " "))
        write(obj_file, string(splitted[3], " "))
        write(obj_file, splitted[4])
      end
      write(obj_file, "\n")
    end
    close(f)
  end
  close(obj_file)

  # Removing all tmp files
  for vtx_file in vertices_files
    #rm(string(modelDirectory, "/", vtx_file))
  end

  for fcs_file in faces_files
    #rm(string(modelDirectory, "/", fcs_file))
  end

end

function assignTasks(startInd, endInd, taskArray)
  """
  This function choose the first files to merge
  creating a tree where number of processes is maximized

  startInd: starting index for array subdivision
  endInd: end index for array subdivision
  taskArray: array containing indices of files to merge for first
  """
  if (endInd - startInd == 2)
    push!(taskArray, startInd)
  elseif (endInd - startInd < 2)
    if (endInd % 4 != 0 && startInd != endInd)
      # Stop recursion on this branch
      push!(taskArray, startInd)
    end
    # Stop recursion doing nothing
  else
    assignTasks(startInd, startInd + trunc((endInd - startInd) / 2), taskArray)
    assignTasks(startInd + trunc((endInd - startInd) / 2) + 1, endInd, taskArray)
  end
end

function mergeVerticesFiles(file1, file2, startOffset)
  """
  Support function for merging two vertices files.
  Returns the number of vertices of the merged file

  file1: path of the first file
  file2: path of the second file
  startOffset: starting face offset for second file
  """

  f1 = open(file1, "a")

  f2 = open(file2)
  debug("Merging ", file2)
  number_of_vertices = startOffset
  for ln in eachline(f2)
    write(f1, ln)
    number_of_vertices += 1
  end
  close(f2)

  close(f1)

  return number_of_vertices
end


function mergeFacesFiles(file1, file2, facesOffset)
  """
  Support function for merging two faces files

  file1: path of the first file
  file2: path of the second file
  facesOffset: offset for faces
  """

  f1 = open(file1, "a")

  f2 = open(file2)
  for ln in eachline(f2)
    splitted = split(ln)
    write(f1, "f ")
    write(f1, string(parse(splitted[2]) + facesOffset, " "))
    write(f1, string(parse(splitted[3]) + facesOffset, " "))
    write(f1, string(parse(splitted[4]) + facesOffset, "\n"))
  end
  close(f2)

  close(f1)
end

function mergeObjProcesses(fileArray, facesOffset = Nothing)
  """
  Merge files on a single process

  fileArray: Array containing files that will be merged
  facesOffset (optional): if merging faces files, this array contains
    offsets for every file
  """

  if(contains(fileArray[1], string("_vtx.stl")))
    # Merging vertices files
    offsets = Array(Int, 0)
    push!(offsets, countlines(fileArray[1]))
    vertices_count = mergeVerticesFiles(fileArray[1], fileArray[2], countlines(fileArray[1]))
    rm(fileArray[2]) # Removing merged file
    push!(offsets, vertices_count)
    for i in 3: length(fileArray)
      vertices_count = mergeVerticesFiles(fileArray[1], fileArray[i], vertices_count)
      rm(fileArray[i]) # Removing merged file
      push!(offsets, vertices_count)
    end
    return offsets
  else
    # Merging faces files
    mergeFacesFiles(fileArray[1], fileArray[2], facesOffset[1])
    rm(fileArray[2]) # Removing merged file
    for i in 3 : length(fileArray)
      mergeFacesFiles(fileArray[1], fileArray[i], facesOffset[i - 1])
      rm(fileArray[i]) # Removing merged file
    end
  end
end

function mergeObjHelper(vertices_files, faces_files)
  """
  Support function for mergeObj. It takes vertices and faces files
  and execute a single merging step

  vertices_files: Array containing vertices files
  faces_files: Array containing faces files
  """
  numberOfImages = length(vertices_files)
  taskArray = Array(Int, 0)
  assignTasks(1, numberOfImages, taskArray)

  # Now taskArray contains first files to merge
  numberOfVertices = Array(Int, 0)
  tasks = Array(RemoteRef, 0)
  for i in 1 : length(taskArray) - 1
    task = @@spawn mergeObjProcesses(vertices_files[taskArray[i] : (taskArray[i + 1] - 1)])
    push!(tasks, task)
    #append!(numberOfVertices, mergeObjProcesses(vertices_files[taskArray[i] : (taskArray[i + 1] - 1)]))
  end

  # Merging last vertices files
  task = @@spawn mergeObjProcesses(vertices_files[taskArray[length(taskArray)] : end])
  push!(tasks, task)
  #append!(numberOfVertices, mergeObjProcesses(vertices_files[taskArray[length(taskArray)] : end]))


  for task in tasks
    append!(numberOfVertices, fetch(task))
  end

  debug("NumberOfVertices = ", numberOfVertices)

  # Merging faces files
  tasks = Array(RemoteRef, 0)
  for i in 1 : length(taskArray) - 1

    task = @@spawn mergeObjProcesses(faces_files[taskArray[i] : (taskArray[i + 1] - 1)],
                                    numberOfVertices[taskArray[i] : (taskArray[i + 1] - 1)])
    push!(tasks, task)

    #mergeObjProcesses(faces_files[taskArray[i] : (taskArray[i + 1] - 1)],
    #                  numberOfVertices[taskArray[i] : (taskArray[i + 1] - 1)])
  end

  #Merging last faces files
  task = @@spawn mergeObjProcesses(faces_files[taskArray[length(taskArray)] : end],
                                  numberOfVertices[taskArray[length(taskArray)] : end])

  push!(tasks, task)
  #mergeObjProcesses(faces_files[taskArray[length(taskArray)] : end],
  #                    numberOfVertices[taskArray[length(taskArray)] : end])

  for task in tasks
    wait(task)
  end

end

function mergeObjParallel(modelDirectory)
  """
  Merge stl files in a single obj file using a parallel
  approach. Files will be recursively merged two by two
  generating a tree where number of processes for every
  step is maximized
  Actually use of this function is discouraged. In fact
  speedup is influenced by disk speed. It could work on
  particular systems with parallel accesses on disks

  modelDirectory: directory containing models
  """

  files = readdir(modelDirectory)

  # Appending directory path to every file
  files = map((s) -> string(modelDirectory, "/", s), files)

  # While we have more than one vtx file and one faces file
  while(length(files) != 2)
    vertices_files = files[find(s -> contains(s,string("_vtx.stl")), files)]
    faces_files = files[find(s -> contains(s,string("_faces.stl")), files)]

    # Merging files
    mergeObjHelper(vertices_files, faces_files)

    files = readdir(modelDirectory)
    files = map((s) -> string(modelDirectory, "/", s), files)
  end

  mergeVerticesFiles(files[2], files[1], 0)
  mv(files[2], string(modelDirectory, "/model.obj"))
  rm(files[1])

end

function mergeAndRemoveDuplicates(firstPath, secondPath)
  """
  Merge two boundary files removing common faces between
  them

  firstPath, secondPath: Prefix of paths to merge
  """

  firstPathV = string(firstPath, "_vtx.stl")
  firstPathFV = string(firstPath, "_faces.stl")

  secondPathV = string(secondPath, "_vtx.stl")
  secondPathFV = string(secondPath, "_faces.stl")

  if(isfile(firstPathV) && isfile(secondPathV))

    V = Array(Array{Int}, 0)
    FV = Array(Array{Int}, 0)

    offset = 0

    # First of all open files and retrieve LAR models

    f1_V = open(firstPathV)
    f1_FV = open(firstPathFV)

    for ln in eachline(f1_V)
      splitted = split(ln)
      push!(V, [parse(splitted[2]), parse(splitted[3]), parse(splitted[4])])
      offset += 1
    end

    for ln in eachline(f1_FV)
      splitted = split(ln)
      push!(FV, [parse(splitted[2]), parse(splitted[3]), parse(splitted[4])])
    end

    close(f1_V)
    close(f1_FV)

    f2_V = open(secondPathV)
    f2_FV = open(secondPathFV)

    for ln in eachline(f2_V)
      splitted = split(ln)
      push!(V, [parse(splitted[2]), parse(splitted[3]), parse(splitted[4])])
    end

    for ln in eachline(f2_FV)
      splitted = split(ln)
      push!(FV, [parse(splitted[2]) + offset, parse(splitted[3]) + offset, parse(splitted[4]) + offset])
    end

    close(f2_V)
    close(f2_FV)
    
    V_final, FV_final = LARUtils.removeVerticesAndFacesFromBoundaries(V, FV)
    
    # Writing model to file
    rm(firstPathV)
    rm(firstPathFV)
    rm(secondPathV)
    rm(secondPathFV)
    writeToObj(V_final, FV_final, firstPath)
  end  
end

function mergeBoundaries(modelDirectory,
                         imageHeight, imageWidth, imageDepth,
                         imageDx, imageDy, imageDz)
  """
  Merge boundaries files. For every cell of size
  (imageDx, imageDy, imageDz) in the model grid,
  it merges right faces with next left faces, top faces
  with the next cell bottom faces, and front faces
  with the next cell back faces

  modelDirectory: directory containing models
  imageHeight, imageWidth, imageDepth: images sizes
  imageDx, imageDy, imageDz: sizes of cells grid
  """

  beginImageStack = 0
  endImage = beginImageStack

  tasks = Array(RemoteRef, 0)
  for zBlock in 0:(imageDepth / imageDz - 1)
    startImage = endImage
    endImage = startImage + imageDz
    for xBlock in 0:(imageHeight / imageDx - 1)
      for yBlock in 0:(imageWidth / imageDy - 1)

        # Merging right Boundary
        firstPath = string(modelDirectory, "/right_output_", xBlock, "-", yBlock, "_", startImage, "_", endImage)
        secondPath = string(modelDirectory, "/left_output_", xBlock, "-", yBlock + 1, "_", startImage, "_", endImage)
        task1 = @@spawn mergeAndRemoveDuplicates(firstPath, secondPath)

        # Merging top boundary
        firstPath = string(modelDirectory, "/top_output_", xBlock, "-", yBlock, "_", startImage, "_", endImage)
        secondPath = string(modelDirectory, "/bottom_output_", xBlock, "-", yBlock, "_", endImage, "_", endImage + 2)
        task2 = @@spawn mergeAndRemoveDuplicates(firstPath, secondPath)

        # Merging front boundary
        firstPath = string(modelDirectory, "/front_output_", xBlock, "-", yBlock, "_", startImage, "_", endImage)
        secondPath = string(modelDirectory, "/back_output_", xBlock + 1, "-", yBlock, "_", startImage, "_", endImage)
        task3 = @@spawn mergeAndRemoveDuplicates(firstPath, secondPath)
        
        push!(tasks, task1, task2, task3)

      end
    end
  end
  
  # Waiting for tasks
  for task in tasks
    wait(task)
  end
end
end
@}


@O src/pngStack2Array3dJulia.jl
@{module PngStack2Array3dJulia

"""
This module loads a stack of png files returning
an array of pixel values divided into segments
"""

export calculateClusterCentroids, pngstack2array3d, getImageData, convertImages

using Images # For loading png images
using Colors # For grayscale images
using PyCall # For including python clustering
using Logging
@@pyimport scipy.ndimage as ndimage
@@pyimport scipy.cluster.vq as cluster

NOISE_SHAPE_DETECT=10

function getImageData(imageFile)
  """
  Get width and heigth from a png image
  """

  input = open(imageFile, "r")
  data = readbytes(input, 24)

  if (data[2:4] != [80, 78, 71] && data[13:16] != [73, 72, 68, 82])
    error("This is not a png image")
  end

  w = data[17:20]
  h = data[21:24]

  width = reinterpret(Int32, reverse(w))[1]
  height = reinterpret(Int32, reverse(h))[1]

  close(input)

  return width, height
end

function calculateClusterCentroids(path, image, numberOfClusters = 2)
  """
  Loads an image and calculate cluster centroids for segmentation

  path: Path of the image folder
  image: name of the image
  numberOfClusters: number of desidered clusters
  """
  imageFilename = string(path, image)

  img = imread(imageFilename) # Open png image with Julia Package

  rgb_img = convert(Image{ColorTypes.RGB}, img)
  gray_img = convert(Image{ColorTypes.Gray}, rgb_img)
  imArray = raw(gray_img)

  imageWidth = size(imArray)[1]
  imageHeight = size(imArray)[2]

  # Getting pixel values and saving them with another shape
  image3d = Array(Array{Uint8,2}, 0)

  # Inserting page on another list and reshaping
  push!(image3d, imArray)
  pixel = reshape(image3d[1], (imageWidth * imageHeight), 1)

  # Segmenting image using kmeans
  # https://en.wikipedia.org/wiki/Image_segmentation#Clustering_methods

  centroids,_ = cluster.kmeans(pixel, numberOfClusters)

  return centroids

end


function pngstack2array3d(path, minSlice, maxSlice, centroids)
  """
  Import a stack of PNG images into a 3d array

  path: path of images directory
  minSlice and maxSlice: number of first and last slice
  centroids: centroids for image segmentation
  """

  # image3d contains all images values
  image3d = Array(Array{Uint8,2}, 0)

  debug("maxSlice = ", maxSlice, " minSlice = ", minSlice)
  files = readdir(path)

  for slice in minSlice : (maxSlice - 1)
    debug("slice = ", slice)
    imageFilename = string(path, files[slice + 1])
    debug("image name: ", imageFilename)
    img = imread(imageFilename) # Open png image with Julia Package

    # Converting image in grayscale
    rgb_img = convert(Image{ColorTypes.RGB}, img)
    gray_img = convert(Image{ColorTypes.Gray}, rgb_img)
    imArray = raw(gray_img) # Putting pixel values into RAW 3d array
    debug("imArray size: ", size(imArray))

    # Inserting page on another list and reshaping
    push!(image3d, imArray)

  end

  # Removing noise using a median filter and quantization
  for page in 1:length(image3d)

    # Denoising
    image3d[page] = ndimage.median_filter(image3d[page], NOISE_SHAPE_DETECT)

    # Image Quantization
    debug("page = ", page)
    debug("image3d[page] dimensions: ", size(image3d[page])[1], "\t", size(image3d[page])[2])
    pixel = reshape(image3d[page], size(image3d[page])[1] * size(image3d[page])[2] , 1)
    qnt,_ = cluster.vq(pixel,centroids)

    # Reshaping quantization result
    centers_idx = reshape(qnt, size(image3d[page],1), size(image3d[page],2))
    #centers_idx = reshape(qnt, size(image3d[page]))

    # Inserting quantized values into 3d image array
    tmp = Array(Uint8, size(image3d[page],1), size(image3d[page],2))

    for j in 1:size(image3d[1],2)
      for i in 1:size(image3d[1],1)
        tmp[i,j] = centroids[centers_idx[i,j] + 1]
      end
    end

    image3d[page] = tmp

  end

  return image3d
end

function convertImages(inputPath, outputPath, bestImage)
  """
  Get all images contained in inputPath directory
  saving them in outputPath directory in png format.
  If images have one of two odd dimensions, they will be resized
  and if folder contains an odd number of images another one will be
  added

  inputPath: Directory containing input images
  outputPath: Temporary directory containing png images
  bestImage: Image chosen for centroids computation

  Returns the new name for the best image
  """

  imageFiles = readdir(inputPath)
  numberOfImages = length(imageFiles)
  outputPrefix = ""
  for i in 1: length(string(numberOfImages)) - 1
    outputPrefix = string(outputPrefix,"0")
  end

  newBestImage = ""
  imageNumber = 0
  for imageFile in imageFiles
    img = imread(string(inputPath, imageFile))

    # resizing images if they do not have even dimensions
    dim = size(img)
    if(dim[1] % 2 != 0)
      debug("Image has odd x; resizing")
      xrange = 1: dim[1] - 1
    else
      xrange = 1: dim[1]
    end

    if(dim[2] % 2 != 0)
      debug("Image has odd y; resizing")
      yrange = 1: dim[2] - 1
    else
      yrange = 1: dim[2]
    end

    img = subim(img, xrange, yrange)

    outputFilename = string(outputPath, outputPrefix[length(string(imageNumber)):end], imageNumber,".png")
    imwrite(img, outputFilename)

    # Searching the best image
    if(imageFile == bestImage)
      newBestImage = string(outputPrefix[length(string(imageNumber)):end], imageNumber,".png")
    end

    imageNumber += 1
  end

  # Adding another image if they are odd
  if(numberOfImages % 2 != 0)
    debug("Odd images, adding one")
    bestImage = imread(string(outputPath, "/", newBestImage))
    imArray = zeros(Uint8, size(bestImage))
    img = grayim(imArray)
    outputFilename = string(outputPath, "/", outputPrefix[length(string(imageNumber)):end], imageNumber,".png")
    imwrite(img, outputFilename)
  end

  return newBestImage
end

end
@}




%===============================================================================
\subsection{Installing the library}
%===============================================================================

%===============================================================================
\section{Conclusions}\label{sec:conclusions}
%===============================================================================
%-------------------------------------------------------------------------------
\subsection{Results}
%-------------------------------------------------------------------------------

%-------------------------------------------------------------------------------
\subsection{Further improvements}
%-------------------------------------------------------------------------------

%-------------------------------------------------------------------------------
\bibliographystyle{amsalpha}
\bibliography{ImagesToLARModel}
%-------------------------------------------------------------------------------
%===============================================================================
\appendix
\section{Utility functions}
%===============================================================================

%-------------------------------------------------------------------------------

\section{Tests}\label{sec:tests}

\paragraph{Generation of the border matrix}
%-------------------------------------------------------------------------------
@O test/generateBorderMatrix.jl
@{push!(LOAD_PATH, "../../")
import GenerateBorderMatrix
import JSON
using Base.Test

function testComputeOriented3Border()
  """
  Test function for computeOriented3Border
  """
  boundaryMatrix = GenerateBorderMatrix.computeOriented3Border(2,2,2)

  rowcount = boundaryMatrix[:shape][1]
  @@test rowcount == 36
  colcount = boundaryMatrix[:shape][2]
  @@test colcount == 8
  row = boundaryMatrix[:indptr]
  @@test row == [0,1,2,3,4,5,7,8,9,11,12,13,15,17,18,19,20,22,23,24,26,27,29,30,32,34,35,37,39,41,42,43,44,45,46,47,48]
  col = boundaryMatrix[:indices]
  @@test col == [0,0,0,1,1,0,1,1,2,0,2,2,3,1,3,2,3,3,2,3,0,4,4,4,1,5,5,4,5,5,2,6,4,6,6,3,7,5,7,6,7,7,6,7,4,5,6,7]
  data = boundaryMatrix[:data]
  @@test data == [-1,1,-1,-1,1,1,-1,1,-1,-1,1,-1,-1,-1,1,1,-1,1,-1,-1,1,-1,1,-1,1,-1,1,1,-1,1,1,-1,-1,1,-1,1,-1,-1,1,1,-1,1,-1,-1,1,1,1,1]

end

function testWriteBorder()
  """
  Test for writeBorder
  """
  boundaryMatrix = GenerateBorderMatrix.computeOriented3Border(2,2,2)
  filename = "borderFile"

  GenerateBorderMatrix.writeBorder(boundaryMatrix, filename)
  @@test isfile(filename)

  # Loading borderMatrix from json file
  borderData = JSON.parsefile(filename)
  row = Array(Int64, length(borderData["ROW"]))
  col = Array(Int64, length(borderData["COL"]))
  data = Array(Int64, length(borderData["DATA"]))

  @@test borderData["ROW"] == [0,1,2,3,4,5,7,8,9,11,12,13,15,17,18,19,20,22,23,24,26,27,29,30,32,34,35,37,39,41,42,43,44,45,46,47,48]
  @@test borderData["COL"] == [0,0,0,1,1,0,1,1,2,0,2,2,3,1,3,2,3,3,2,3,0,4,4,4,1,5,5,4,5,5,2,6,4,6,6,3,7,5,7,6,7,7,6,7,4,5,6,7]
  @@test borderData["DATA"] == [-1,1,-1,-1,1,1,-1,1,-1,-1,1,-1,-1,-1,1,1,-1,1,-1,-1,1,-1,1,-1,1,-1,1,1,-1,1,1,-1,-1,1,-1,1,-1,-1,1,1,-1,1,-1,-1,1,1,1,1]

  rm(filename)

end

function executeAllTests()
  @@time testComputeOriented3Border()
  @@time testWriteBorder()
  println("Tests completed.")
end

executeAllTests()

@}
%-------------------------------------------------------------------------------

\paragraph{Conversion of a png stack to a 3D array}
%-------------------------------------------------------------------------------
@O test/pngStack2Array3dJulia.jl
@{push!(LOAD_PATH, "../../")
import PngStack2Array3dJulia
using Base.Test

function testGetImageData()
  """
  Test function for getImageData
  """

  width, height = PngStack2Array3dJulia.getImageData("images/0.png")

  @@test width == 50
  @@test height == 50

end

function testCalculateClusterCentroids()
  """
  Test function for calculateClusterCentroids
  """
  path = "images/"
  image = 0
  centroids = PngStack2Array3dJulia.calculateClusterCentroids(path, image, 2)

  expected = [0, 253]
  centroids = vec(reshape(centroids, 1, 2))

  @@test sort(centroids) == expected
end

function testPngstack2array3d()
  """
  Test function for pngstack2array3d
  """
  path = "images/"
  minSlice = 0
  maxSlice = 4
  centroids = PngStack2Array3dJulia.calculateClusterCentroids(path, 0, 2)
  image3d = PngStack2Array3dJulia.pngstack2array3d(path, minSlice, maxSlice, centroids)

  @@test size(image3d)[1] == 5
  @@test size(image3d[1])[1] == 50
  @@test size(image3d[1])[2] == 200

end

function executeAllTests()
  @@time testCalculateClusterCentroids()
  @@time testPngstack2array3d()
  @@time testGetImageData()
  println("Tests completed.")
end

executeAllTests()

@}
%-------------------------------------------------------------------------------

\paragraph{Test for LAR utilities}
%-------------------------------------------------------------------------------
@O test/LARUtils.jl
@{push!(LOAD_PATH, "../../")
import LARUtils
using Base.Test

function testInd()
  """
  Test function for ind
  """

  nx = 2
  ny = 2

  @@test LARUtils.ind(0, 0, 0, nx, ny) == 0
  @@test LARUtils.ind(1, 1, 1, nx, ny) == 13
  @@test LARUtils.ind(2, 5, 4, nx, ny) == 53
  @@test LARUtils.ind(1, 1, 1, nx, ny) == 13
  @@test LARUtils.ind(2, 7, 1, nx, ny) == 32
  @@test LARUtils.ind(1, 0, 3, nx, ny) == 28
end

function executeAllTests()
  @@time testInd()
  println("Tests completed.")
end

executeAllTests()

@}
%-------------------------------------------------------------------------------


\end{document}
